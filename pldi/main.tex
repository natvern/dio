%%
%% This is file `sample-acmsmall-conf.tex',
%% generated with the docstrip utility.
%%
%% The original source files were:
%%
%% samples.dtx  (with options: `acmsmall-conf')
%% 
%% IMPORTANT NOTICE:
%% 
%% For the copyright see the source file.
%% 
%% Any modified versions of this file must be renamed
%% with new filenames distinct from sample-acmsmall-conf.tex.
%% 
%% For distribution of the original source see the terms
%% for copying and modification in the file samples.dtx.
%% 
%% This generated file may be distributed as long as the
%% original source files, as listed above, are part of the
%% same distribution. (The sources need not necessarily be
%% in the same archive or directory.)
%%
%%
%% Commands for TeXCount
%TC:macro \cite [option:text,text]
%TC:macro \citep [option:text,text]
%TC:macro \citet [option:text,text]
%TC:envir table 0 1
%TC:envir table* 0 1
%TC:envir tabular [ignore] word
%TC:envir displaymath 0 word
%TC:envir math 0 word
%TC:envir comment 0 0
%%
%%
%% The first command in your LaTeX source must be the \documentclass
%% command.
%%
%% For submission and review of your manuscript please change the
%% command to \documentclass[manuscript, screen, review]{acmart}.
%%
%% When submitting camera ready or to TAPS, please change the command
%% to \documentclass[sigconf]{acmart} or whichever template is required
%% for your publication.
%%
%%
\documentclass[acmsmall]{acmart}

%%
%% \BibTeX command to typeset BibTeX logo in the docs
\AtBeginDocument{%
  \providecommand\BibTeX{{%
    Bib\TeX}}}


%%
%% Submission ID.
%% Use this when submitting an article to a sponsored event. You'll
%% receive a unique submission ID from the organizers
%% of the event, and this ID should be used as the parameter to this command.
%%\acmSubmissionID{123-A56-BU3}

%%
%% For managing citations, it is recommended to use bibliography
%% files in BibTeX format.
%%
%% You can then either use BibTeX with the ACM-Reference-Format style,
%% or BibLaTeX with the acmnumeric or acmauthoryear sytles, that include
%% support for advanced citation of software artefact from the
%% biblatex-software package, also separately available on CTAN.
%%
%% Look at the sample-*-biblatex.tex files for templates showcasing
%% the biblatex styles.
%%

%%
%% The majority of ACM publications use numbered citations and
%% references.  The command \citestyle{authoryear} switches to the
%% "author year" style.
%%
%% If you are preparing content for an event
%% sponsored by ACM SIGGRAPH, you must use the "author year" style of
%% citations and references.
%% Uncommenting
%% the next command will enable that style.
%%\citestyle{acmauthoryear}


\usepackage{bussproofs}
\usepackage{proof}
\usepackage{algorithm}
\usepackage[noend]{algpseudocode}
\usepackage[toc]{glossaries}
\usepackage{glossary-mcols}
\usepackage{soul}

\theoremstyle{definition}
\newtheorem{definition}{Definition}[section]
\loadglsentries{terminology}

% For comment boxes.
\usepackage[colorinlistoftodos,prependcaption,textsize=tiny]{todonotes}
\newcommand{\giselle}[1]{\todo[linecolor=red,backgroundcolor=red!25,bordercolor=red]{G: #1}}
\newcommand{\samar}[1]{\todo[linecolor=blue,backgroundcolor=blue!25,bordercolor=blue]{S: #1}}
\newcommand{\dio}{\textsc{dio}}

%%
%% end of the preamble, start of the body of the document source.
\begin{document}


%%
%% The "title" command has an optional parameter,
%% allowing the author to define a "short title" to be used in page headers.
\title{Domain Informed Oracle for Reinforcement Learning}

%%
%% The "author" command and its associated commands are used to define
%% the authors and their affiliations.
%% Of note is the shared affiliation of the first two authors, and the
%% "authornote" and "authornotemark" commands
%% used to denote shared contribution to the research.
\author{Samar Rahmouni}
\email{srahmoun@andrew.cmu.edu}
\orcid{0000-0003-1351-1515}
\affiliation{%
  \institution{Carnegie Mellon University}
  \city{Doha}
  \country{Qatar}
}

\author{Giselle Reis}
\email{giselle@cmu.edu}
\orcid{0000-0002-5145-9829}
\affiliation{%
  \institution{Carnegie Mellon University}
  \city{Doha}
  \country{Qatar}
}


%% By default, the full list of authors will be used in the page
%% headers. Often, this list is too long, and will overlap
%% other information printed in the page headers. This command allows
%% the author to define a more concise list
%% of authors' names for this purpose.
%\renewcommand{\shortauthors}{Trovato et al.}


%%
%% The abstract is a short summary of the work to be presented in the
%% article.
\begin{abstract}
  %% The story

  %% There is this great new method!
  Reinforcement learning (RL) is a powerful AI
  technique that does not
  require pre-gathered data but relies on a trial-and-error process
  for the agent to learn. 
  %
  This is made possible through a reward function that associates
  state configurations to a numerical value. 
  %
  The agent's goal is to maximize its cumulative reward over its
  lifetime. 

  %% Oh, but there are some issues...
  Unfortunately, there is no systematic method to design a reward
  function since interpreting abstract states in RL in
  the context of a domain needs to be done on a case by case basis.

  %% This issue is crucial [motivation]
  \textcolor{red}{Summary of motivation.}
  
  %% We can fix that!
  We propose a \emph{Domain Informed Oracle (\dio{})} to
  systematically incorporate domain specific knowledge into RL
  reward functions.
  %
  \dio{} is a collection of domain specific rules written in a
  declarative language, such as Prolog.
  %
  It does not rely on the RL representation of states, allowing the
  programmer to focus on the domain knowledge using an
  expressive and intuitive language, where states and rules can
  be defined conveniently.
  %
  For each state and action pair, \dio{} provides
  information to the reward
  function, to dynamically adapt itself.
  %
  %% Methodology
  Our implementation is tested on a grid world with dynamic obstacles.
  and compared to a basic RL algorithm. 
  %
  %% Results and Conclusions
  \textcolor{red}{ADD results and conclusions.}
  

\end{abstract}

%%
%% The code below is generated by the tool at http://dl.acm.org/ccs.cfm.
%% Please copy and paste the code instead of the example below.
%%
\begin{CCSXML}
  <ccs2012>
     <concept>
         <concept_id>10003752.10003790.10003795</concept_id>
         <concept_desc>Theory of computation~Constraint and logic programming</concept_desc>
         <concept_significance>300</concept_significance>
         </concept>
   </ccs2012>
\end{CCSXML}
  
\ccsdesc[300]{Theory of computation~Constraint and logic programming}

%%
%% Keywords. The author(s) should pick words that accurately describe
%% the work being presented. Separate the keywords with commas.
\keywords{probabilistic logic programming, reinforcement learning, reward shaping}
%% A "teaser" image appears between the author and affiliation
%% information and the body of the document, and typically spans the
%% page.

%%\received{20 February 2007}
%%\received[revised]{12 March 2009}
%%\received[accepted]{5 June 2009}

%%
%% This command processes the author and affiliation and title
%% information and builds the first part of the formatted document.
\maketitle

\section{Introduction}

Implementing a robust adaptive controller that is effective in terms
of precision, time, and quality of decision in dynamic and uncertain
scenarios has always been a central challenge in AI and robotics. When
autonomous agents are deployed in the real world, we want to ensure
that they are able to adapt to unforeseen scenarios, as well as keep
their efficiency. This efficiency is measured in terms of optimality
of actions and time to make a decision. 
%
Since we are unable to
provide a repertoire of all possible scenarios and actions, agents
need to be able to autonomously predict and adapt to changes.
Reinforcement Learning (RL) is an approach that supports dynamically
adapting to new input~\cite{sutton2018reinforcement}. It has beed
successfully used by AlphaGo, Deepmind AlphaStar, and OpenAI Five to
solve Go, StarCraft II and Dota 2~\cite{li2019reinforcement}.
%
Reinforcement Learning is a powerful tool as it does not require
pre-gathered data as most Machine Learning (ML) techniques do. 
%
The general idea of RL is learning via trial-and-error, guided by a
\textit{domain dependent} reward function.
%
For example, if the agent is a self-driving car, the reward function
would greatly penalize states where it crashes.
%
However, this means that the car is bound to crash to learn not to
crash again.
%
A better reward function can include the physics equations to predict,
with some degree of certainty, the car's trajectory for the next few
seconds.
%
By looking into the future, the reward function can penalize bad
behavior before it reaches a catastrophic state (a crash).
%
A better reward function prunes the (often infinite) search space
faster, allowing the agent to explore (breadth) new states instead of
exploiting (depth) dead ends.
%
The task of choosing the reward function is thus crucial, yet difficult as shown in \ref{sec:challenges}.\samar{Adding a reference to motivation instead of emphasizing here.}
%
In this work, we propose a Domain Informed Oracle (\dio{}) written in a
declarative language to inform a reinforcement learning algorithm. 
%
Our method provides a systematic way to encode domain specific rules
into a reward function for RL that does not rely on the state
representation within the RL algorithm.

%%
%% Our motivation from the lack of symbolic reasoning
%% to inform a reinforcement learning modules
%% The challenges of finding a 'good' reward function
%%
\section{Motivation} 

%% Before beginning, let's explain what reinforcement learning is
Reinforcement Learning is a method of learning that maps situations to
actions in order to maximize its rewards
\cite{sutton2018reinforcement}. Rewards are numerical values associated to a state and action. Precisely, one defines a reward function 
$R : (S \times A) \rightarrow \mathbb{R}$ where $S$ defines the state space and $A$ the action space. Note that a state refers to the current configuration
of the environment and the action refers to the action chosen by the RL agent. By defining this reward function and the scenario of the problem the agent is trying to solve, 
reinforcement learning has the advantage of not requiring a prior dataset. Indeed, the agent is not told what to do, but rather 
learns from the effect of its actions on the environment. 

\begin{figure}[H]
  \centering
  \includegraphics[scale=0.4]{figures/rlroutine.png}
  \caption{Reinforcement Learning Routine}
  \label{fig:rl}
\end{figure}


The diagram in Figure~\ref{fig:rl} is a high-level description of how
an agent using reinforcement learning can be trained.  
%
The upper left box represents the \emph{environment} as seen by the agent
according to its sensors.
%
The current state of the environment is represented as a \emph{state
vector}.
%
At each iteration, the agent will receive the state vector as input,
and needs to choose an \emph{action} to take.
%
Once the action is taken, the environment is updated to the next state
and the agent receives a \emph{reward} as feedback.
%
This reward is a domain dependent function that represents how
``good'' the new state is.
%
The agent's goal is to increase its reward by taking actions that
reach better states each time.
%
The triple (state, action, reward) helps the agent in shaping the
final policy.

%% Reward function is crucial for RL
The reward function is a crucial aspect of the RL algorithm.
%% Here's an example of how it affects training
For instance, consider a game of chess 
where the agent is punished when it loses and rewarded if it wins. The agent is bound to learn how to 
maximize its winnings but it will need to exhaust multiple possible combinations to learn. In this case, 
the training time is not optimal. A better approach would be to also reward it for making a good opening, for instance. 
Another example would be only considering negative rewards. Say we want our agent to escape a maze, and we punish it at every timestep for not escaping. 
If there is a fatality state (\emph{e.g.}, a fire or a black whole), the agent will learn to move towards the fatality state as to cut its negative rewards as soon as possible. 
In conclusion, a good reward function is the first step of optimal learning.
%% Even research says the same! 
By choosing a \emph{refined} reward function, 
we can ensure a faster and more efficient training~\cite{Koenig1996}, possibly with fewer errors. 

\subsection{Reward Shaping}
\label{sec:challenges}

%% Reward shaping is important for a fast and efficient learning
\emph{Reward shaping}~\cite{laud2011} refers to the lack of systematic methods to design a reward
  function that ensures fast and efficient learning~\cite{Koenig1996}. This generation of an appropriate 
reward function for a given problem is still an open challenge~\cite{kober2013}.
%
%% Look at the evidence that shows the importance of it! (examples)
The importance of a reward function for efficient training is shown in~\cite{Koenig1996}.
A sparse reward function defined as a \emph{goal-reward representation} is one where the agent is only rewarded 
for entering a goal state, while a dense reward function is defined as an \emph{action-penalty representation}, precisely, one 
where the agent is penalized for every action it executes. \cite{Koenig1996} shows that a denser reward structure improves performance. 
%
%% That's not all! A good reward function can guide exploration and exploitation as necessary 
Moreover, an informed reward function is able to sufficiently deter the exploration of 
undesirable states while encourae the exploitation of desirable ones, continuously adapting to 
acquire knowledge and resolving the conflict when necessary. Precisely, an informed reward function 
can tackle the \emph{exploration vs. exploitation dilemma}, a central challenge in RL~\cite{Kaelbling1996ReinforcementLA}. 
%
%% But when deployed, it can also help in tackling new/unforeseen situations better
In particular, an agent that can continuously acquire knowledge is responding to the uncertainties of the world, where states can never be exhausted. 
This involves an adaptive learned policy that responds to conditions and tasks that were not encountered in the past~\cite{gupta_meta-reinforcement_2018,schweighofer_meta-learning_2003}.
%

%% 'Perfect' reward function would be a native reward
Ideally, rewards would be given by the real-world, i.e. \textit{native rewards}. For instance, recent work investigates dynamically generating a reward 
using a user verbal feedback to the autonomous agent~\cite{gonzalez2010}.
%
%% However, this cannot be applied in practice because most training happens in simulation
%
However, most RL agents can only stay in simulation because the trial-and-error nature of RL prevents any guarantees of safety.
Thus, there exists a need for \textit{shaping rewards} instead.
%
%% As a result, people can only use those 'sparse' reward structure
Unfortunately, due to the lack of systematic method to build a denser reward function, 
it is common to use a sparse reward function as long as it still guarantees convergence.

%%
%% What is it that we propose? 
%% Domain Informed Oracle! A logic programming module to 
%% inform a reinforcement learning module through reward-shaping
%%
\section{Domain Informed Oracle} 
\subsection{Architecture}
In this section, we lay the foundations of the architecture that combines the Domain Informed Oracle with 
reinforcement learning. Note that in our proposed architecture, we suppose Proximal Policy Optimization, a specific method to compute the policy in RL that is explained in more details in \ref{sec:rldetails}. 
It does not mean that our solution is specific to it, rather it can be generalized to any algorithm choice.

\medskip 

\begin{figure}[H]
  \centering
  \begin{minipage}{.5\textwidth}
    \centering
    \includegraphics[width=1\linewidth]{figures/basicrl.png}
    \captionof{figure}{Reinforcement learning architecture}
    \label{fig:basicrl}
  \end{minipage}%
  \begin{minipage}{.45\textwidth}
    \centering
    \includegraphics[width=1\linewidth]{figures/dio.png}
    \captionof{figure}{\dio{}+RL architecture}
    \label{fig:diorl}
  \end{minipage}
\end{figure}

The diagram in Figure \ref{fig:basicrl} describes the basic routine of RL in more details. The environment defined by the scenario sends the current state 
to the \emph{Proximal Policy Optimization} algorithm. More information on the algorithms used found in \ref{sec:rldetails}. The agent chooses an action from the action space and sends it to the environment. This action affects the environment stepping it to some next state. 
The resulting state along its associated reward is computed from the reward function and step function formalized in the scenario. Thus, in the next iteration, the 
agent receives the reward from its previous action which it uses to improve its policy and continues with its training starting from the computed next state.  

\medskip 

The architecture in Figure \ref{fig:diorl} introduces \dio{} in the feedback loop. It is kept independent of the RL module. Precisely, when the scenario is query-ed for the reward and 
the resulting next state of a (state, action) pair, rather than computing the reward using the reward function, the latter is able to query \dio{}. The result of this query is $J$, a judgment which we keep 
obscure. The fundamental idea is that $J$ is used to inform the reward function when it is tasked with computing the reward. 
 

\subsection{\dio{} procedure}
\label{sec:modules}
In practice, we consider the following modules and their interactions as shown in \ref{fig:mods}.


\begin{multicols}{2}
\begin{enumerate}
  \item \textbf{World Rules} defining the rules governing the world. This is domain-dependent and implemented 
        in a logic programming file, i.e. we are able to define the next step via step semantics.
  \item \textbf{Knowledge Base} defining the ground facts which describe the world at a given time step. This module is 
        continuously updated to account for the dynamics of the
        state.
  \item \textbf{Labels} i.e., textual ``norms'' corresponding to an
  iteration of the state. In practice, they are all possible judgments on the resulting state, e.g. \textit{crash :- obs(X,Y), agent(X,Y)}, 
                  or \textit{maybecrash :- nextObs(X,Y), agent(X,Y)}. 
                    Those labels have probabilities associated with
                    them.
  \item \textbf{Translation Unit} defining the translation from state to ground facts and from labels to a numerical value, e.g. if the predicate crash is true with $P = 0.25$, then the reward shaped is $r + -0.25$. 
  \item \textbf{Reinforcement Learning} is our independent module that does not make assumption on the algorithm chosen for RL.
\end{enumerate}
\end{multicols}


\begin{figure}[H]
  \centering
  \includegraphics[scale=0.4]{figures/dynamics.png}
  \caption{Modules \& Interactions}
  \label{fig:mods}
\end{figure}

Figure \ref{fig:mods} describes the interactions of the different modules, basically taking a closure look to \dio{}. Precisely, 
given the rules and the world knowledge base at a given time $t$, we are able 
to produce the corresponding label, i.e. the query over a predicate. The predicate is fed 
into the Translation Unit (TU) that transforms the predicate to a numerical value that is given to the Reinforcement Learning 
as a reward shaping $r'$. This new reward can either \emph{overwrite} the previous reward, or \emph{fine-tune it}. In general, 
the way this reward affects the initial reward is a \textbf{design choice} that we leave for future investigations. 
Finally, as a result of the RL action, the next state is given 
to TU that translates it into Ground facts to update the world
knowledge, thus restarting the loop. Note that the inference on the query is done by a declarative tool that incorporates 
probabilities called \emph{Problog} that we introduce in \ref{sec:problog}.


\subsection{Problog Procedure} 
\label{sec:problog}
% High level overview of Problog
Problog is a logic programming language that aims to bridge between probabilistic 
logic programming and statistical relational learning \cite{fierens_van}. 
A problog program specifies a probability distribution over possible worlds. 
This probability distribution corresponds to the possible worlds whether a fact is taken 
or discarded given the probability associated with it. Precisely, they define a world 
is a subset of ground probabilistic facts where the probability of the subset is the product of 
the probabilities of the facts it contains.

\paragraph{Statistical Relational Learning (SRL)}
    Discipline of Artificial Intelligence that considers first order logic relations between 
    structures of a complex system and model it through probabilistic graphs such as Bayesian or 
    Markov networks.

\paragraph{Probabilistic Logic Programming (PLP)}
    Discipline of Programming Languages that augments traditional logic programming such as Prolog 
    with the power to infer over probabilistic facts to support the modeling of structured 
    probability distributions.


% Evidence and inference tasks in Problog
Furthermore, problog extends PLP with the power of considering evidences 
in the inference task. This is made possible without requiring the transformation 
the Bayesian networks on which to use SRL. Instead, problog considers the subset described above 
and assumes only worlds where the evidence held remains true. Those possible worlds and their associated 
probabilities are then added and divided by the choice with the higher probability. Problog makes this 
possible by a 3-steps conversion from a problog program to a weighted boolean formula.

% Conversion steps to weighted formula
First, problog grounds the program by only considering facts relevant to the query in question. 
The relevant ground rules are specifically converted to equivalent Boolean formulas. 
Precisely, inferences are converted into bi-directional implications and its corresponding premises 
are converted to a conjunction of disjunction of facts. 
Finally, problog asserts the evidence by adding it to the previous boolean formula 
as a conjunction and defines a weight function that assigns a weight to every literal. 
The weights are derived from the probability associated with the relevant literal, whether explicility 
given or implicility computed. 



%%
%% How would this look in practice?
%% Let's go through our working example. 
%%
\section{Dynamic Obstacles in a GridWorld} 
We first evaluate the performance of our dio/rl implementation compared to an implementation making only use of rl. 
\subsection{Scenario in Reinforcement Learning}
Our autonomous agent exists in an 8x8 grid world. Its goal is to reach the goal from his initial position (1,1).
Along the way, there exists dynamic obstacles which movements is unknown. The agent is punished if colliding with an obstacle and the episode, hereby ends. 
This environment offered by gym-gridworld \cite{gym_minigrid} is useful for testing our algorithm in a Dynamic Obstacle avoidance for a partially observable 
environment. Precisely, we define the state as follows. 
\begin{equation*}
  S_t = [x, y, d, G]
\end{equation*}
$(x,y)$ define the position of our agent while $d$ its direction. $G$ is the gridworld observed by the agent which includes walls, obstacles and free squares. 
The action space is, 
\begin{equation*}
  A_t = \{ right: 0, up: 1, down: 2, left: 3 \}
\end{equation*}
Finally, the reward is a function of the distance from the goal defined as, 
\begin{equation*}
  R_t = 1 - 0.9*(\dfrac{steps}{max\_steps})
\end{equation*}

\medskip

\begin{multicols}{2}
    \begin{figure}[H]
      \centering
      \includegraphics[scale=0.55]{figures/gridworldrl.png}
      \caption{Gridworld with Dynamic Obstacles}
      \label{fig:gridrl}
    \end{figure}
    \columnbreak
    The Reinforcement Learning experiments have been performed on 1M frames for a similar start configuration as shown in \ref{fig:gridrl}. The episode ends when the agent 
    reaches the goal OR collides with an obstacle. We want to encourage the shortest and safest path, thus, the punishment for crashing is $r = -1$. Our rewards are normalized as shown in
    the reward function. We define the range of rewards to be $(-1,1)$. 
\end{multicols}

\subsection{Domain Specific Rules}
The rules are defined as a ProbLog \cite{problog}: a probabilistic prolog that allows us to capture 
the stochasticity of the environment. Precisely, we want to consider the erratic movements of the obstacles, considering 
we do not have previous knowledge on the distribution of their given movement. We assume a uniform distribution and define the following. 
The rules of DIO take the following form: 

\begin{prooftree}
  \AxiomC{$P_0 :: \varphi(0)$}
  \AxiomC{$P_1 :: \varphi(1)$}
  \AxiomC{$P_2 :: \varphi(2)$}
  \AxiomC{$P_3 :: \varphi(3)$}
  \AxiomC{$\ldots$}
  \RightLabel{(action)}
  \QuinaryInfC{$p_1,\ldots, p_n$}
\end{prooftree}
We define $\sum_{i=0}^{n}P(i) = 1$, and $\varphi(i)$ corresponds to the conjunction of grounds facts of the possible world with probability $P_i$.
The action is equivalent to our step semantics, thus, we enforce that a given action modifies the facts in some form. In practice, an action is the missing 
clause to generate the next predicate. In the gridworld example, we give the following. 

\begin{multicols}{2}
\begin{prooftree}
  \AxiomC{atPos(X + V*T, Y)}
  \RightLabel{(right)}
  \LeftLabel{(1)}
  \UnaryInfC{atPos(X,Y), speed(V), timestep(T)}
\end{prooftree}
\columnbreak 
\begin{prooftree}
  \AxiomC{0.25 :: obs(X + V*T, Y, V) \ldots}
  \LeftLabel{(2)}
  \RightLabel{(time)}
  \UnaryInfC{obs(X,Y,V), timestep(T)}
\end{prooftree}
\end{multicols}

(1) considers the movement of the agent while (2) considers the movement of the obstacles. Note that (2) considers 
a uniform distribution over the movement of the obstacle, since every obstacle has a uniform probability of moving up/down/left/right. 
We could do the same for (1) by consider the probability of an action failing. In our case, we assume the movement is deterministic and no failure over the movement 
of the agent happens.


\subsection{World Knowledge}
Our world knowledge base covers the agent, the obstacles and the timestep. We consider two cases: 
\textit{constant} ground facts vs. \textit{dynamic} ground facts. The latter represents positions which are dynamically 
generated at every timestep while the former considers only the facts that remain true in every world, thus include the timestep, since we always
move by 1-unit, and the speed, since the agent and the obtacles are defined to only move by 1-box every time. Given that our knowledge base $Kb$ is defined by, 
\[
    C = \{speed(1), timestep(1) \}     
    \qquad
    D = \{atPos(X, Y), obs(X,Y,1)\}
    \qquad
    Kb = C \cup D
\]

\subsection{From Norms to Labels}
\textcolor{red}{Todo.}

\subsection{Translation Unit}


%%
%% Methodology to include our metrics of safety and optimality
%% During both training and deployement
%%
\section{Methodology} 

\subsection{Metrics}
We compare our architecture incorporating \dio{} to a reinforcement learning architecture. 
Precisely for our scenario presented in \textcolor{red}{SECTION}, we judge \emph{optimality}, i.e. the number of steps to reach the goal and 
\emph{safety}, i.e. ratio of successes to failures. Those metrics are analyzed during both training and deployement. 
We gather the cumulative rewards and cumulative rewards as well as the total number of failures and success over 180k frames of training, and 1000 episodes of deployement. 

\subsection{Settings} 
Our above metrics and gathered for 8 different settings. We consider both $\alpha$, 0 (equivalent to rl alone), 0.5, 1, 3 and 9. 
Similarly, we consider three settings for the number of obstacles: (1) minimal (1/20 of the board is covered by obstacles), (2) intermediate (1/10th) and finally, 
(3) extreme (1/3rd). For each, we gather the metrics described above. 

\subsection{Gathered Results} 
\paragraph{During Training.} For optimization purposes, training is done using multiprocessing reinforcement learning as indicated in figure~\ref{fig:multiprocess}. 
The network is loaded and parallel environments are spawned given the available number of processes. For $n$ frames, the rl routine is returned before the final experience is collected 
and used to update the parameters of the neural network. For our purpose, the cumulative reward is the reward of each $(obs_i, r_i)$ added over the number of updates. 
Similarly, the terminal states are computed according to this similar philosophy, meaning that at every given environment, if the resulting state is terminal and a failure, we add it to the cumulative failures to assess 
safety overtime. At the end of training, we compute the ratio of successes over failures to analyze the needed number of failures before convergence.

\begin{figure}[H]
    \centering
    \includegraphics[scale=0.6]{figures/multiprocess.png}
    \caption{Multi-processing Reinforcement Learning}
    \label{fig:multiprocess}
  \end{figure}
  \paragraph{During Deployement.} Once training converges and a resulting policy is computed, we deploy the given policy for a 1000 episodes, given an episode ends either 
  with a success or a failure. We compute the average number of steps it takes to reach the goal and its standard deviation to assess the optimality of the resulting paths. 
  For safety, we still consider the ratio of success over failures as an indication. 
  

%%
%% Results given alpha and obstacles settings 
%%
\section{Results}
\textcolor{red}{The FINAL PART AAAA.}

%% 
%% Related work 
%%
\section{Related Work}
%
%% Our inspiration : Neurosymbolic AI! What is it?
%
To tackle the reward shaping challenge from Section~\ref{sec:challenges}, we are
inspired by the current Neurosymbolic AI trends, which explore
combinations of deep learning (DL) and symbolic reasoning.
%
The work has been a response to criticism on DL's lack of formal
semantics and intuitive explanation, and the lack of expert knowledge
towards guiding machine learning models.

%
%% How do they incorporate domain knowledge?
%
Current Neurosymbolic AI trends are concerned with knowledge representation and reasoning, namely, they investigate computational-logic systems 
and representation to precede learning in order to provide some form
of incremental update, e.g. a meta-network to group two sub-neural
networks~\cite{Besold2017NeuralSymbolicLA}.
As a result, neurosymbolic AI has been successfully applied to vision-based tasks such as semantic labeling \cite{vinyals2015, karpathy2015}, 
vision analogy-making \cite{Reed2015DeepVA}, or learning communication
protocols \cite{Foerster2016LearningTC}.
%
%% When they do, there are promising results in machine learning.
%
In general, neurosymbolic AI trends show promising results in improving ML algorithms, whether that is from 
an interpretability aspect or an optimization one. More recent works take this trend and incorporate symbolic reasoning and 
domain knowledge in reinforcement learning settings \cite{Driessens2010,Romero2020,achiam2017,marek2010}. \cite{marek2010,Romero2020} use the general idea of \textit{reward shaping} and \textit{epsilon adaptation} respectively 
to incorporate procedural knowledge into a RL algorithm. 
Both works introduce this combination as a successful strategy to guide the exploration and exploitation tradeoff in RL. They both show promising results. While 
\cite{marek2010} focuses on providing formal specifications for reward shaping, it lacks practical 
consequences to the implementation of most RL to make use of its formal methods conclusions. On the other hand, \cite{Romero2020} proposes a method to adapt $\epsilon$ based on domain knowledge, the method is specifically applied to "Welding Sequence Optimization".  
To do so, the RL algorithm is modified in itself, similarly to what was done in \cite{Driessens2010}. Precisely, in \cite{Driessens2010}, the RL algorithm itself is 
modified to deal with states that are model-based as opposed to vectors. They defined their method as Relational RL. 
Furthermore, they conclude that by using more expressive representation language for the RL scenario, their method can potentially offer a solution to the problem of meta-learning. 
While \cite{Romero2020,Driessens2010} both present promising rewards, they lack the modularity necessary for scaling the proposed methods to further RL implementations. 

%
%% But this is ALSO the case for reinforcement learning specifically: reward machines
%
This is further reinforced by more recent work, precisely, \emph{reward machines} that define an automaton to adapt a reward function given 
step transitions~\cite{icarte2022reward}. By exposing the structure in the reward function, \cite{icarte2022reward} shows that this enables to find solutions faster. 
%
%% This is great! But there's an issue with reward machines..
%
However, given the nature of a state machine, reward machines are unable to adapt to the uncertainties of the world. 

%
%% And thus, we provide dio..
%
To face those limitations, \dio{} does not rely on an abstract representation to infer a reward function, rather only needs 
to care about the translation to a domain specific language, like prolog or datalog to assess a given world. 
The stochasticity of the world is then inherent, given a probabilistic logic program.
\dio{} provides a more declarative approach to reason about rewards, thus providing a systematic method to map 
labels to rewards. 

 

\section{Conclusions}

In conclusion, as RL faces the issues of reward shaping, meta-learning and the exploration-exploitation dilemma, domain knowledge show promising results in 
improving reinforcement learning methods. The main challenge is to make such an integration seamless, and independent of the AI implementation. 
This is a task we were able to produce in our simpler introductary example of the Dynamic Obstacles in the grid world. Results are promising and will be further 
extended to the traffic simulation referred to in \ref{traffic}. More directions open up as we think of optimization, this includes mix-matching the n-steps approach with the 
number of steps \dio{} can look ahead. Similarly, we look at the differences between overwriting vs. fine-tuning the rewards using \dio{} and if such choice matters in training. 
As we start adding complexity to the algorithm, we turn our focus into the specifications as shown in \ref{scspecs} to better inform 
on meaningful and effective way to translate our labels to their corresponding numerical values design choice. 


%%
%% The acknowledgments section is defined using the "acks" environment
%% (and NOT an unnumbered section). This ensures the proper
%% identification of the section in the article metadata, and the
%% consistent spelling of the heading.

% TODO: acknowledge reviewers
%\begin{acks}
%To Robert, for the bagels and explaining CMYK and color spaces.
%\end{acks}

%%
%% The next two lines define the bibliography style to be used, and
%% the bibliography file.
\bibliographystyle{ACM-Reference-Format}
\bibliography{biblio.bib}


%%
%% If your work has an appendix, this is the place to put it.
%\appendix


\end{document}
%\endinput
%%
%% End of file `main.tex'.
