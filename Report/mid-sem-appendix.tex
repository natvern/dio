\documentclass[a4paper,11pt]{article}

\usepackage{amssymb}
\usepackage{amstext}
\usepackage{amsmath}
\usepackage{amsthm}
\usepackage{booktabs}
\usepackage{graphicx}
\usepackage{float}
\usepackage{url}
\usepackage{proof}

\theoremstyle{definition}
\newtheorem{definition}{Definition}[section]

\usepackage{xcolor}
\newcommand{\bluenote}[1]{\color{blue}{ \em #1 }\color{black}}

\usepackage{geometry}
 \geometry{
 a4paper, %letterpaper,
 total={17cm,22.8cm},
 margin=24mm,
 top=22.4mm,
 bottom=25.4mm
 }
 
% Times new roman
\usepackage{mathptmx}

\usepackage[T1]{fontenc}

\usepackage{caption}
\usepackage{subcaption}

\thispagestyle{empty} 

\setlength{\parindent}{0pt}

% For comment boxes.
\usepackage[colorinlistoftodos,prependcaption,textsize=tiny]{todonotes}
\newcommand{\giselle}[1]{\todo[linecolor=red,backgroundcolor=red!25,bordercolor=red]{G: #1}}
\newcommand{\samar}[1]{\todo[linecolor=blue,backgroundcolor=blue!25,bordercolor=red]{S: #1}}

\author{%
%  \fontsize{10}{12}\selectfont
  \begin{minipage}[t]{0.47\textwidth}
    \centering
    Samar Rahmouni \\ srahmoun@andrew.cmu.edu
  \end{minipage}
  \and
  %
  \begin{minipage}[t]{0.45\textwidth}
    \centering
    Advisor: Prof. Giselle Reis \\ giselle@cmu.edu
  \end{minipage}%
  \vspace*{2ex}
}


\date{}

\title{{\Large\sc Senior Thesis 2021-22\\[2ex]}{\LARGE\bf Domain Informed Oracle (DIO) for Reinforcement Learning\vspace*{3ex}}}


\begin{document}

\maketitle 

\begin{abstract} 

  %% The story

  %% There is this great new method!
  Reinforcement learning (RL) is a powerful AI method that does not
  require pre-gathered data but relies on a trial-and-error process
  for the agent to learn. 
  %
  This is made possible through a reward function that associates
  current state configurations to a numerical value. 
  %
  The agent's goal is then to maximize its cumulative reward over its
  lifetime. 
  %
  %% Oh, but there are some issues...
  Unfortunately, RL suffers multiple challenges: (1) reward shaping,
  (2) meta-learning and (3) the exploration-exploitation dilemma. 
  %
  (1) refers to the lack of systematic methods to design a reward
  function that ensures fast and efficient learning.
  %
  (2) is the problem of deploying an agent trained in a simulation to
  the real-world, or possibly another simulation, where it encounters
  state configurations it did not during its training. The goal is to
  be able to efficiently adapt to those configurations.
  %
  Finally, (3) is the question of whether to always exploit what the
  agent knows or explore in the hope that an unexplored state might
  result in better rewards. 

  \medskip

  % This is needed to connect the ideas from the previous paragraph
  % and the next.

  %% No problem, we can solve them!
  These challenges can be alleviated for a given domain by fine
  tuning the reward function with domain specific information. 
  %
  For example, if the agent is a self-driving car, the reward function
  can include the physics equations to predict, with some degree of
  certainty, the car's trajectory for the next few seconds.
  %
  By looking into the future, the reward function can penalize bad
  behavior before it reaches a catastrophic state (\emph{e.g.}, a
  crash).
  %
  A better reward function prunes the (often infinite) search space
  faster, allowing the agent to explore (breadth) new states instead
  of exploiting (depth) dead ends.
  %
  %% Oh, but only on a case by case basis...
  However, tuning the reward might be hard depending on the domain and
  how states are represented.
  %
  States are typically represented as vector of values in RL, and
  translating properties and rules from a domain into this
  representation can be complicated depending on how many values are
  used, what they represent, whether they are normalized or not, etc.

  \medskip

  %% We can fix that!
  We propose a \emph{Domain Informed Oracle (DIO)} as a solution for
  systematically incorporating domain specific knowledge into RL
  reward functions.
  %
  DIO is a collection of domain specific rules and judgments written
  in a declarative language, such as Prolog.
  %
  It does not rely on the RL representation of states, allowing
  the programmer to focus on the domain specific knowledge using an
  expressive and intuitive language, where they can define states and
  rules in the most convenient way.
  %
  DIO provides an informed decision to the reward function, thus
  allowing it to dynamically adapt the rewards. 
  %
  %DIO is incorporated into a RL architecture and is defined by its
  %behavior in relation to the input given by the reward function. 
  %
  Our implementation is tested on a Traffic Simulator scenario and
  compared to a basic uninformed RL algorithm. 
  %
  The comparison is based on performance which we define by three
  metrics: time to train, optimality of the learned policy and
  finally, number of errors through training. 
  
\end{abstract}



\section{Introduction}
Implementing a robust adaptive controller that is effective in terms
of precision, time, and quality of decision
when facing dynamic and uncertain scenarios, has always been a central
challenge in AI and robotics. Precisely, as we want our autonomous agents to be deployed 
in the real world, we want to ensure that they are able to adapt to unforeseen scenarios, as well as 
keep their efficiency. This efficiency is measured in terms of their optimality and time taken to produce a decision. 
%
As autonomous cars are deployed, IoT is popularized, and human-robot interactions become more complex, we
are more and more confronted with the need for robotic agents that can effectively and continually adapt
to their surroundings, not only in simulation, but also in practice, when deployed as a cyber-physical system. 
Since we are unable to provide a repertoire of all possible scenarios and actions,
our agents need to be able to autonomously predict and adapt to new
changes. Reinforcement Learning (RL) is an approach that
supports dynamically adapting to new input. It is also the solution that AlphaGo, Deepmind AlphaStar, and OpenAI Five have
adopted \cite{li2019reinforcement}, respectively for Go, StarCraft II and Dota 2 and found success in. 

\medskip

Reinforcement Learning is a powerful tool as it does not require pre-gathered data as most Machine Learning (ML) 
techniques do. The general idea of RL is a trial-and-error process. In other words, a car trained using RL is bound to 
crash to learn not to crash again. This learning is based on a \textit{domain dependent} reward function. For example, suppose we would like to teach an autonomous car to stop at a STOP sign. We will 
punish it via a negative reward when it does not, and reward it if it does. Thus, through multiple iterations, 
and with the goal to maximize its rewards, a reinforcement learning trained agent will learn to stop at a STOP sign. 
%

\medskip

The task of choosing a reward function that ensures optimality is thus crucial. In the following work, we propose a Domain Informed Oracle (DIO) to
inform a reinforcement learning . Precisely, we propose a systematic way to encode domain specific rules to inform a reward function for RL. We argue that such a combination will ensure 
a faster and more efficient RL trained agent in terms of optimality. The proposed combination is then tested on a traffic simulation. The results are compared with 
a RL implementation that makes use of standard practices to design a reward function. 



\section{Reinforcement Learning} 
Reinforcement Learning is a method of learning that maps situations to
actions in order to maximize its rewards
\cite{sutton2018reinforcement}. Rewards are numerical values associated to a state and action. Precisely, one defines a reward function 
$R : (S * A) \rightarrow \mathbb{R}$ where $S$ defines the state space and $A$ the action space. Note that a state refers to the current configuration
of the environment and the action refers to the action chosen by the RL agent. By making use of this reward function and the trial-and-error process, 
reinforcement learning has the advantage of not requiring a prior dataset. Indeed, the agent is not told what to do, but rather 
learns from the effect of its actions on the environment. Consider figure \ref{fig:rl}. 
\begin{figure}[H]
  \centering
  \includegraphics[scale=0.6]{rlroutine.png}
  \caption{Reinforcement Learning Routine}
  \label{fig:rl}
\end{figure}


The diagram in Figure~\ref{fig:rl} is a high-level description of how
an agent using reinforcement learning can be trained.  
%
The lower box represents the \emph{environment} as seen by the agent
according to its sensors.
%
The current state of the environment is represented as a \emph{state
vector}.
%
At each iteration, the agent will receive the state vector as input,
and needs to choose an \emph{action} to take.
%
Once the action is taken, the environment is updated to the next state
and the agent receives a \emph{reward} as feedback.
%
This reward is a domain dependent function that represents how
``good'' the state is.
%
The agent's goal is to increase its reward by taking actions that
reach better states each time.\giselle{This last sentence is like a
magic box to me. Maybe we need to make it less magic, maybe not.}\samar{I think it's fine to 
keep it as a black box. It will involve mentioning Bayesian and Markov decision process and 
I didn't need those in the latter half.} 
%

\medskip 

To emphasize again, the reward function is a crucial aspect of the RL algorithm. For instance, consider a game of chess 
where the agent is punished when it loses and rewarded if it wins. The agent is bound to learn how to 
maximize its winnings but it will need to exhaust multiple possible combinations to learn how to play. In this case, 
the training time is not optimal. A better approach would be to additionally reward it for making a good opening. 
Another example of a bad reward choice would be only considering negative rewards. Say we want our agent to escape a maze, and we punish it at every timestep for not escaping. 
If there exists a fatality state, the agent will learn to move towards the fatality state as to cut its negative rewards as soon as possible. 
In conclusion, a good reward function is the first step towards optimal learning. By choosing the right reward function, 
we can ensure a faster and more efficient training, possibly with fewer errors. 

\section{Challenges in Reinforcement Learning}

\subsection{Reward Shaping}
The task of choosing a reward function that ensures optimality is crucial. However, the generation of an appropriate 
reward function for a given problem is still an open challenge \cite{kober2013}. This problem is known as \textit{Reward Shaping} \cite{laud2011}.  
Ideally, rewards would be given by the real-world, i.e. \textit{native rewards}. For instance, recent work investigates dynamically generating a reward 
using a user verbal feedback to the autonomous agent \cite{gonzalez2010}. However, most RL agents 
can only stay in simulation due to the lack of safety guarantees. This is because of the trial-and-error nature of the RL training. 
Thus, there exists a need for \textit{shaping rewards} instead. There are reasonable criteria on how this should be done, those are \emph{standard practices}. For instance, 
rewards that can be continuously harvested speed up convergence compared to rewards that can only be harvested at 'the end' (i.e. the chess example). Similarly, one should avoid only providing negative 
rewards as that results in unwanted behavior (i.e. the maze example). Furthermore, if dealing with a continuous state space, a polynomial differential reward function 
is shown to help the agent learn faster. Finally, one can normalize rewards at the end as to not end up with too many discrepancies. Unfortunately, these standard practices do not always ensure 
an optimal reward function and the problem of reward shaping is reduced to a trial-and-error process. In other words, one arbitrarily updates their reward function until the learned policy approaches optimal. 
Therefore, there still exists a lack of a systematic method to design a reward function.

\subsection{Exploration vs. Exploitation}
A central issue in reinforcement learning is the dilemma of \emph{exploration} vs. \emph{exploitation}. Consider this problem. An agent is at an intersection. It has the choice of going either right or left. 
It does not yet know the outcome of either. It chooses right at a given point and receives a reward $r=1$. The question is "When faced with the same decision, should it keep going right?" There are two issues to consider. 
First, it does not know the outcome of going left. It could be that there is a better reward waiting for it on the left lane. Second, when dealing with a stochastic environment, it might be that $r$ was a one-time occurrence. 
It would be equivalent to someone buying a lottery ticket, and winning \$1M on their first try, and later, deciding to spend all their winnings in trying to make it happen again. This problem showcases the importance of exploration; an agent 
needs to see where other paths might lead to, but also exploitation; if it keeps exploring forever it will never accumulate rewards. This is especially evident when the possible states cannot be exhausted. Several techniques have been proposed 
to balance between exploration and exploitation \cite{Kaelbling1996ReinforcementLA}. A notable one is the \emph{epsilon-greedy} technique. The idea is to set some probability $\epsilon$ by which the agent decides to explore. This probability can be adapted 
to decrease as more \emph{episodes} are completed. 

\subsection{Meta-Learning}
The result of training is a policy $\pi$, a function to map states to action $\pi : S \rightarrow A$. The learned policy is the one that maximizes the cumulative rewards.
This training is most often done in simulation, given the lack of safety guarantees of RL. 
However, several problems come into place when considering the deployment of the trained agent. Considering that an agent has done well in 
a designated simulation does not imply that it will do as well in the real-world. Overall, it must be that certain uncertainties will not be expected, thus there can be no expectation on how the agent will behave 
when out of simulation. This problem of adaptation is known as \emph{Meta-learning}. Meta-learning in reinforcement learning is the problem of learning-to-learn, which is about efficiently
adapting a learned policy to conditions and tasks that were not encountered in the past. In RL, meta-learning
involves adapting the learning parameters, balancing exploration and exploitation to direct the
agent interaction \cite{gupta_meta-reinforcement_2018,schweighofer_meta-learning_2003}. Meta-learning is a central problem in AI, since an agent that can solve more
and more problems it has not seen before, approaches the ideal of a general-purpose AI. 


\section{Symbolic Reasoning for Reinforcement Learning} \label{symrl}
As to tackle the challenges from section 3, we are inspired from current Neurosymbolic AI trends. Precisely, neurosymbolic integration has been an ongoing work in the last years towards a combination of deep learning (DL) and symbolic reasoning.
The work has been a response to criticism of DL on its lack of formal semantics and intuitive explanation and the lack of expert knowledge towards guiding machine learning models. 
Key questions the field targets are identifying the necessary and sufficient building blocks of AI \cite{garcez2020neurosymbolic}, namely, how can we provide the semantics of knowledge, 
and work towards meta-learning? 

Current Neurosymbolic AI trends are concerned with knowledge representation and reasoning, namely, they investigate computational-logic systems 
and representation to precede learning in order to provide some form of incremental update, e.g. a meta-network to group two sub-neural networks. \cite{Besold2017NeuralSymbolicLA}
This leads to neurosymbolic AI finding various applications including vision-based tasks such as semantic labeling \cite{vinyals2015, karpathy2015}, 
vision analogy-making \cite{Reed2015DeepVA}, or learning communication protocols \cite{Foerster2016LearningTC}. Those results inspire us to use those techniques for reinforcement learning, as to tackle its challenges.

\medskip
Rewards are domain dependent and thus, given domain specific rules, a \emph{domain informed} module can guide a RL agent towards better decisions. This can be done by 
adapting the reward function. For instance, we consider defining which states are desirable, which are to be avoided and which are fatal. Given rules and judgments, a logic programming module 
is able to search the space and send feedback to the reinforcement learning agent. The goal is a systematic method to design a reward function which can ensure faster and more efficient 
training. This knowledge can furthermore be incorporated into resolving the exploration vs. exploitation dilemma. For instance, if a domain informed module 
can infer that only one of the possible next states is desirable, then exploration in that specific case is suboptimal.  
We will call the proposed module a \emph{Domain Informed Oracle (DIO)}. 

\begin{figure}[H]
  \centering
  \includegraphics[scale=0.5]{overview.png}
  \caption{Overview of the proposed solution}
  \label{fig:overview}
\end{figure}

The diagram in Figure~\ref{fig:overview} is a high-level description of our proposed solution. 
%
The box on the left represents a basic reinforcement learning algorithm that depends on the scenario of the given problem and 
the common standard practices discussed previously to design a reward function. 
%
The box on the right represents our proposed domain knowledge to inform the reinforcement learning algorithm. 
Precisely, the domain informed oracle is given the scenario and can thus start a feedback loop between itself and 
the informed RL module to update the rewards. 
%
Finally, those two implementations will be compared based on their performance. In the following, we define performance 
given three metrics: (1) time to train (2) optimality of the learned policy and (3) number of errors through training. 
Consider 'errors' as suboptimal decisions that were made by the agent while in the process of training. For example, exploring a (state, action) pair 
that has previously given a negative reward is suboptimal. 

\section{Domain Informed Oracle} 
\subsection{Architecture}
In this section, we lay the foundations of the architecture that combines the Domain Informed Oracle with 
reinforcement learning. Note that in our proposed architecture, we suppose Q-learning, a specific method to compute the policy in RL. 
It does not mean that our solution is specific to Q-learning. 

\medskip 

\begin{figure}[H]
  \centering
  \includegraphics[scale=0.41]{basicrl.png}
  \caption{Reinforcement learning architecture}
  \label{fig:basicrl}
\end{figure}


\begin{figure}[H]
  \centering
  \includegraphics[scale=0.42]{dio.png}
  \caption{Dio+RL architecture}
  \label{fig:diorl}
\end{figure}

The diagram in Figure \ref{fig:basicrl} describes the basic routine of RL in more details. The environment defined by the scenario sends the current state 
to the Q-Learning algorithm. The agent chooses an action from the action space and sends it to the environment. This action affects the environment stepping it to some next state. 
The resulting state along its associated reward is computed from the reward function and step function formalized in the scenario. Thus, in the next iteration, the 
agent receives the reward from its previous action which it uses to improve its policy and continues with its training starting from the computed next state.  

\medskip 

The architecture in Figure \ref{fig:diorl} introduces DIO in the feedback loop. It is kept independent of the RL module. Precisely, when the scenario is query-ed for the reward and 
the resulting next state of a (state, action) pair, rather than computing the reward using the reward function, the latter is able to query DIO. The result of this query is $A$ which we keep 
obscure. The fundamental idea is that $A$ is used to inform the reward function when it is tasked with computing the reward. 

\subsection{Specifications} \label{scspecs}
DIO is a logic programming based module that takes the query from the reward function and returns $A$, a judgment that the scenario awaits to update its reward function. The judgment is obscured as it is a choice of the scenario, 
independent of the DIO implementation. In the following, we will define $A$ by its behavior, rather than its type. To do so, we will walk through the routine in the diagram of Figure \ref{fig:diospecs}.


\begin{figure}[H]
  \centering
  \includegraphics[scale=0.46]{diospecs.png}
  \caption{Domain Informed Oracle routine}
  \label{fig:diospecs}
\end{figure}

\begin{enumerate}
  \item The scenario queries DIO. It sends $(s_t, a)$, the state at time $t$ and the action. It also sends $L : s \rightarrow t$ where $t$ is a sum of types: 
        $t \equiv t_1 + t_2 .. + t_n$. For instance, consider our previous example of the chess game and let the scenario define good openings as "desirable" and random openings as 
        "undesirable". Thus $t \equiv d + ud$ where $d$ is equivalent to desirable and $ud$ to undesirable. 
  \item The first step of DIO is a translation $T : \mathbb{R}^n \rightarrow t_1 \times t_2 \times ... t_n$ that takes in the state vector $s_t$ and returns a conjunction of propositions, let it be $P$. $P$ is equivalent to 
        our set of ground facts.
  \item P is passed alongside the action to the domain specific rules defined in DIO. 
  \item Using step semantics, DIO generates possible worlds up to a given time $t'$. Note that given the stochasticity of the environment, there is not certainty on
        whether the worlds expected by DIO will necessarily happen. Consider Figure \ref{fig:diosim}.
  \item Those possible worlds, equivalent to states, can be transformed using $L$ to $t$. 
  \item At the last step, DIO ends up with a set $S$ of $t$ that defines the labels of all the possible worlds.
  \item The last step is left as an inference depending on how the the judgment for deciding a final $A$ from $S$ 
        is formulated. For instance, we can consider a rule where if most of the possible worlds are undesirable, then let $A$ be undesirable.
  \item Finally, $A$ is passed to the scenario. Note then that $A : t$. 
\end{enumerate}

\begin{figure}[H]
  \centering
  \includegraphics[scale=0.55]{scworlds.png}
  \caption{Game Tree Simulation by DIO}
  \label{fig:diosim}
\end{figure}

Observe how this does not guarantee that the reward function will be informed in a meaningful way.
To that end, we need to consider (1) what are the specifications of a 'good' reward function, (2) what conditions should be set 
on the domain specific rules and $L$ provided by the scenario, (3) how to define a systematic way for the scenario to use 
this information. Those considerations are left for future investigations. 
 

\section{Traffic Simulation} \label{traffic}
Given the definition of our DIO architecture, the performance as shown in Figure \ref{fig:overview} will be judged on a Traffic Simulation. 
Precisely, we make use of the OpenSource code of Carla \cite{Dosovitskiy17} to put our implementation in practice. 
\subsection{Scenario in Reinforcement Learning}
Our autonomous agent is one vehicle on the road. The road is populated with other vehicles, walkers and traffic signs and lights. 
The goal of the agent is to get from point $A$ to point $B$ on the map in the fastest time possible. Carla defines \emph{sensors} to collect 
data from the world. From the sensors, we collect \textbf{collision detector}, \textbf{obstacle detector}, \textbf{IMU} (which defines the internal state of the vehicle that includes acceleration)
and finally \textbf{position} data to incorporate into our state representation. Thus, 
\begin{equation*}
  S_t = [t, p, c, o, a]
\end{equation*} 
In other words, the state at time $t$ is defined by $t$ the time, $p$ the position and $a$ the acceleration. $c$ and $o$ are both flags (-1/1) that indicate 
respectively whether there is a collision or an obstacle detected. (-1) is equivalent to detected while (1) is equivalent to non-detected. Furthermore, 
\begin{equation*}
  A_t = ](a_t)-min, (a_t)+max[
\end{equation*}
Precisely, our action space at time $t$ is a range delimited by the minimum acceleration possible and the maximum acceleration possible. Finally, 
\begin{equation*}
  R_t(s_t) = c * (d(i, p)/t) + (o * a)
\end{equation*}
We define $d$ a function that takes in the initial position $i$ and the current position $p$ and computes the distance travelled. 
By putting everything together, we can see that the reward is maximized when the agent travels the longest distance in a shortest 
amount of time (without colliding at high speed) while making sure that when an obstacle is detected, acceleration is kept low. 
\subsection{Domain Specific Rules in DIO}
Note that DIO is a logic programming module that we design in Prolog and 
carry to Python via PySwip \cite{tekol2020}.
The first step of DIO is to consider the translation. 
\[
  \infer[T_1]{[t, p, -1, 1, a]}{\text{Time}(t) \wedge \text{Pos}(p) 
                                \wedge \text{Acc}(a) \wedge
                              \text{Collision}}
   \qquad
   \infer[T_2]{[t, p, -1, -1, a]}{\text{Time}(t) \wedge \text{Pos}(p) 
                                  \wedge \text{Acc}(a) \wedge
                                  \text{Collision} \wedge \text{Obstacle}}
\]

\[
  \infer[T_3]{[t, p, 1, 1, a]}{\text{Time}(t) \wedge \text{Pos}(p) 
                                \wedge \text{Acc}(a)
                            }
   \qquad
   \infer[T_4]{[t, p, 1, -1, a]}{\text{Time}(t) \wedge \text{Pos}(p) 
                                  \wedge \text{Acc}(a) \wedge \text{Obstacle}}
\]

The domain specific rules will follow the same idea, taking in the ground facts and the action, 
and infering the next set of ground facts. Thus, DIO searchs for configurations of ground facts given the step semantics, 
to return the set of labels. The main challenge is pruning the search space as we want to 
guarantee time efficiency. While we hypothesize that the RL would learn faster by incorporating 
domain knowledge, we are expecting iterations to take longer as the feedback loop incorporates DIO's computation. This part is an ongoing work. 

\section{Related Work}
As discussed previously in section \ref{symrl}, neurosymbolic AI trends show promising results in improving ML algorithms, whether that is from 
an interpretability aspect or an optimization one. More recent works take this trend and incorporate symbolic reasoning and 
domain knowledge in reinforcement learning settings \cite{Driessens2010,Romero2020,achiam2017,marek2010}. \cite{marek2010,Romero2020} use the general idea of \textit{reward shaping} and \textit{epsilon adaptation} respectively 
to incorporate procedural knowledge into a RL algorithm. Both introduce this combination as a successful strategy to guide the exploration and exploitation tradeoff in RL. They both show promising results. While 
\cite{marek2010} focuses on providing formal specifications for reward shaping, it lacks practical 
consequences to the implementation of most RL to make use of its formal methods conclusions. On the other hand, \cite{Romero2020} proposes a method to adapt $\epsilon$ based on domain knowledge, the method is specifically applied to "Welding Sequence Optimization".  
To do so, the RL algorithm is modified in itself, similarly to what was done in \cite{Driessens2010}. Precisely, in \cite{Driessens2010}, the RL algorithm itself is 
modified to deal with states that are model-based as opposed to vectors. They defined their method as Relational RL. 
Furthermore, they conclude that by using more expressive representation language for the RL scenario, their method can be potentially offer a solution to the problem of meta-learning. 
While \cite{Romero2020,Driessens2010} both present promising rewards, they lack the modularity necessary for scaling the proposed methods to further RL implementations. Those by taking their results into consideration, 
we are hopeful that symbolic reasoning and RL integration could provide a solution for reward shaping, meta-learning and the exploration-exploitation dilemma. 

\section{Conclusions}
In conclusion, as RL faces the issues of reward shaping, meta-learning and the exploration-exploitation dilemma, domain knowledge show promising results in 
improving reinforcement learning methods. The main challenge is to make such an integration seamless, and independent of the AI implementation. 
As we start incorporating our modules into one architecture to test on our Traffic Simulator in Section \ref{traffic}, we will be able to show preliminary 
results. Similarly, we look further into the specifications of DIO to ensure that it informs the scenario in a meaningful way as discussed in Section \ref{scspecs}. 

\newpage

\bibliographystyle{unsrt}
\bibliography{biblio.bib}


\section{Appendix}

In the following, we describe our previous work that was done towards considering safe reinforcement learning. 
Precisely, we were motivated by the lack of safety guarantees and the use of symbolic reasoning towards incorporating 
safety properties from a verified safe controller (SC) i.e. a declarative language module to compute safe and unsafe states. 
Those safety states are computed during training, thus ensuring that no unsafe states are explored 
by the RL. The general idea was to allow training in the real-world, thus taking away the need for simulation. Our proposed RL+SC architecture is shown in Figure \ref{fig:rlsc}.


\begin{figure}[H]
  \centering
  \includegraphics[scale=0.5]{rlsc.png}
  \caption{Overview of the proposed solution}
  \label{fig:rlsc}
\end{figure}

\begin{enumerate}
  \item The environment sends the observation to the SC. 
  \item The SC computes the state from the observation, thus keeping only 
        ground facts that uphold the \emph{Markov Property} (see Definition 9.1). 
  \item The SC also computes the set of possible actions $A$ that ensure that no unsafe states is 
        reached. This is done by searching for possible configurations that follow a (state, action) pair over 
        all possible actions from the action spaces. 
  \item The state and set of possible actions is sent to the RL algorithm, thus only allowing 
        the agent to choose from the set actions. 
  \item Next steps follow from the RL basic routine from Figure \ref{fig:rl}
\end{enumerate}

The implementation can be found in the thesis corresponding github under the rl-implementation folder\footnote{https://github.com/natvern/Thesis}.
For our case study, we chose the vehicle platooning problem. The setting is as follows; two vehicles, one leader and one follower. Their goal is to 
minimize the gap between them while ensuring that they do not crash. 

\medskip

We then present our preliminary results before arguing for the issues with our approach. 

\begin{figure}[H]
  \centering
  \begin{minipage}{.5\textwidth}
    \centering
    \includegraphics[width=1\linewidth]{appendixopt.png}
    \captionof{figure}{Cumulative Rewards of RL vs. RL+SC}
    \label{fig:optsc}
  \end{minipage}%
  \begin{minipage}{.5\textwidth}
    \centering
    \includegraphics[width=1\linewidth]{appendixcrash.png}
    \captionof{figure}{Cumulative Crashes of RL during Training}
    \label{fig:crashsc}
  \end{minipage}
\end{figure}
  
The graph in Figure \ref{fig:optsc} shows the cumulative rewards of the basic RL implementation without
the safe controller (blue line) vs. the RL+SC implementation (orange line). As evident in the cumulative rewards, all are negative. This is because of our reward function choice 
that only punishes for distances not equal to the desired gap. Thus, learning is not optimal. We can however see that the cumulative rewards of both implementation does not change by a significant 
amount. The graph in Figure \ref{fig:crashsc} shows 
the number of crashes of the basic RL during training. We can easily deduce from the linear graph that the RL without the SC never learns not to crash.


\begin{definition}[Markov Property]
  The next state only depends on the value of the current state. In other words, only the present can affect the future. 
  In terms of our state vector, this means that given $S_t$, I only need the features in $S_t$ to predict $S_{t+1}$.
\end{definition}

\subsection{Issues with the safe RL approach}
\begin{itemize}
  \item The simplicity of our problem setting does not showcase the need for a SC as we only consider two vehicles on an x-axis. 
  \item Though ensuring safety guarantees during training theoretically takes away the need for a simulation, in practice, the uncertainties 
        of the real world cannot be all predicted. This results in a handful of safety properties that might be guaranteed, not enough to allow sensitive cyber-physical systems 
        to be trained using RL in the real world. 
  \item If an oracle existed that was able to predict all uncertainties of the real world, this oracle could then be used to solve the optimization problem 
        deterministically without the need for RL. 
  \item Depending on the SC implementation, safety guarantees can be too strict thus completely destroying optimality. Consider a SC that only allows a car not to move. 
        Though it guarantees that no crashes happen, it is also too strict, thus going against the goal of the agent. 
  
\end{itemize}




\end{document}
