\section{Optimization} \label{optimality} 

\subsection{Problog Procedure} 
% High level overview of Problog
Problog is a logic programming language that aims to bridge between probabilistic 
logic programming and statistical relational learning \cite{fierens_van}. 
A problog program specifies a probability distribution over possible worlds. 
This probability distribution corresponds to the possible worlds whether a fact is taken 
or discarded given the probability associated with it. Precisely, they define a world 
is a subset of ground probabilistic facts where the probability of the subset is the product of 
the probabilities of the facts it contains.

\paragraph{Statistical Relational Learning (SRL)}
    Discipline of Artificial Intelligence that considers first order logic relations between 
    structures of a complex system and model it through probabilistic graphs such as Bayesian or 
    Markov networks.

\paragraph{Probabilistic Logic Programming (PLP)}
    Discipline of Programming Languages that augments traditional logic programming such as Prolog 
    with the power to infer over probabilistic facts to support the modeling of structured 
    probability distributions.


% Evidence and inference tasks in Problog
\paragraph{}
Furthermore, problog extends PLP with the power of considering evidences 
in the inference task. This is made possible without requiring the transformation 
the Bayesian networks on which to use SRL. Instead, problog considers the subset described above 
and assumes only worlds where the evidence held remains true. Those possible worlds and their associated 
probabilities are then added and divided by the choice with the higher probability. Problog makes this 
possible by a 3-steps conversion from a problog program to a weighted boolean formula.

\paragraph{}
% Conversion steps to weighted formula
First, problog grounds the program by only considering facts relevant to the query in question. 
The relevant ground rules are specifically converted to equivalent Boolean formulas. 
Precisely, inferences are converted into bi-directional implications and its corresponding premises 
are converted to a conjunction of disjunction of facts. 
Finally, problog asserts the evidence by adding it to the previous boolean formula 
as a conjunction and defines a weight function that assigns a weight to every literal. 
The weights are derived from the probability associated with the relevant literal, whether explicility 
given or implicility computed. 


\subsection{Compile once Evaluate many}
Since the grounding of the problog program is dependent on the query, it is possible to first compile the model 
to a problog internal format and ground all possible queries and evidence specified in the model. This allows us 
to compile once and evaluate many. This can help with the optimization of our program, since our files can be defined into 
(1) rules, (2) world knowledge base and (3) labels as explained in \ref{sec:modules}. (1) and (3) are constants, while the dynamics 
of the world (2) change at every given timestep. We can then initially ground the constants and later 
extend the database to incorporate the environment knowledge. 
 
\textcolor{red}{To write.} \samar{mention multiprocessing 
and transformation to evidence based probabilities}

\subsection{Feedback n-steps} 
Another possible optimization is to extend the number of steps for which feedback is requested. In the 
following, we evaluate the tradeoff between optimality and speed of training when increasing $n$. 



